\documentclass[12pt, letterpaper]{memoir}
\usepackage{ExamStyle}

\begin{document}
	\mainmatter
	
	\begin{center}
		{\Huge CHEM 1115}
		{\LARGE-- Fall 2021
		
		Final Exam Study Guide}
	\end{center}
	
	This study guide is meant to be only a guide as you study. Please review the material and questions from the lab manual rather than just relying on this guide. Equations and tables, for example, are found in the lab manual rather than in this guide.
	
	\subsection*{Lab 1: Measurement and Glass-Bending}	
	\begin{itemize}
		\item As a matter of safety, remember that glass cools down slowly and hot glass looks just the same as cool glass
		\item Volume measurements should be taken from the bottom of the meniscus
		\item Volumes can be estimated to one significant figure beyond the graduation marks
	\end{itemize}

	\subsection*{Lab 2: Density}
	\begin{itemize}
		\item Significant figure rules -- see your textbook
		\item Density from mass and dimensional measurements
		\item Measuring density from mass and displacement volume
		\item Layering of immiscible liquids
		\item Density of mixed solutions can be surprisingly high (water/ethanol mixture)
	\end{itemize}
	
	\subsection*{Lab 3: Atomic Identification}
	\begin{itemize}
		\item Quantum mechanical explanation for atomic emission and absorption lines
		\item Identify an atom based on an observed color and a table of absorption/emission colors
	\end{itemize}
	
	\subsection*{Lab 4: Chemical Formulas}
	\begin{itemize}
		\item Difference between empirical and molecular formulas
		\item Determine stoichiometric ratio from yield of precipitation reactions
		\item Determine stoichiometric ratio from gravimetric analysis of an oxidation reaction
	\end{itemize}
	
	\subsection*{Lab 5: Chemical Names and Structures}
	\begin{itemize}
		\item Write chemical formulas from chemical names
		\item Give chemical names from chemical formulas (molecular and ionic)
		\item Identify molecular geometry
		\item Identify polar bonds and net molecular polarity		
	\end{itemize}
	
	\subsection*{Lab 6: Chemical Reactions}
	\begin{itemize}
		\item Solubility rules
		\item How to recognize a chemical reaction
		\item Classifying chemical reactions
	\end{itemize}
	
	\subsection*{Lab 7: Stoichiometry}
	\begin{itemize}
		\item Calculating \% mass of copper in copper compounds based on the chemical formulas
		\item Finding \% mass of copper in a sample by reducing copper ions to copper metal
	\end{itemize}
	
	\subsection*{Lab 8: Chemical Equilibrium and Le Ch\^atelier's Principle}
	\begin{itemize}
		\item Response of a system to addition/removal of a reactant or product
		\item Response of a system to temperature changes (endo/exothermic reactions)
		\item Response of a system to pressure changes
		\item Use of a syringe to change the pressure on a system
	\end{itemize}
	
	\subsection*{Lab 9: Collection of Hydrogen Gas}
	\begin{itemize}
		\item Calculating moles of gas from the ideal gas law
		\item Dalton's law of partial pressures
		\item Vapor pressure of water
		\item Proper use of a eudiometer
		\item \% yield from theoretical and actual yields
	\end{itemize}
	
	\subsection*{Lab 10: Electrolytes}
	\begin{itemize}
		\item Definition of strong and weak electrolytes
		\item Identifying strong and weak electrolytes from conductivity
		\item Identifying strong and weak electrolytes from a gold nanoparticle solution
	\end{itemize}
	
	\subsection*{Lab 11: Acid-Base Chemistry}
	\begin{itemize}
		\item Finding pH from \ch{[H3O^+]}
		\item Finding \ch{[H3O^+]} from pH
		\item Identifying acidity using color indicators (litmus paper and indicator solutions)
		\item Other tests for acidity (Mg and \ch{Ca(OH)2})
		\item Sodium and Chloride qualitative tests
	\end{itemize}
\end{document}