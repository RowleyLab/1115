\documentclass[12pt, letterpaper]{article}
\usepackage{SyllabusStyle}

\begin{document}
\begin{center}
	{\Large \textsc{Elementary Chemistry Lab}}

	CHEM 1115
\end{center}

\begin{center}
	{\large Spring 2022}
\end{center}
\begin{center}
	\rule{0.99\textwidth}{0.4pt}
	\begin{tabular}{llcll}
		\textbf{Instructor:} & Matthew Rowley           &  & \textbf{Office Hours:} & Daily 10:00 am -- 11:00 am \\
		\textbf{Telephone:}  & (435) 586-7875           &  &                        &                            \\
		\textbf{Email:}      & matthewrowley$1$@suu.edu &  & \textbf{Office:}       & SC-220                     \\
		\multicolumn{5}{c}{Please include the course number in the subject line of all correspondence.}
	\end{tabular}
	\rule{0.99\textwidth}{0.4pt}
\end{center}

\section*{Course Description}
This is the lab to accompany CHEM 1110.

\paragraph{Prerequisites:}
None

\paragraph{Concurrent requisite:}
CHEM 1110 -- Elementary Chemistry

\paragraph{Course Materials:} ~

\emph{Experiments for ELEMENTARY CHEMISTRY -- CHEM 1115} by Bronsema (Available at the SUU Bookstore)

You are required to bring and wear your own pair of OSHA-approved safety goggles to \emph{every} lab. Students without eye protection will be required to leave the lab and will receive a zero for the labwork that day.

\paragraph{Student Learning Outcomes:}
\begin{description}
	\item[Knowledge of the physical and natural world] -- Students will recall, interpret, compare, explain, and apply chemistry terminology and theory.
	\item[Quantitative Literacy] -- Students will use chemical equations, graphs and tables to interpret and communicate chemical information.
	\item[Inquiry and Analysis] -- Students will investigate chemical problems.
	\item[Communication] -- Students will report laboratory results clearly and concisely.
	\item[Problem Solving] -- Students will implement experimental procedures.
	\item[Teamwork] -- Students will productively interact with each other to successfully conduct chemistry experiments.
\end{description}

\section*{Laboratory Work}
Before lab, you are expected to have read the handout of your experiment as well as review your lecture notes from class. Come prepared to enter your data into the lab computers and have a USB drive with you. You may perform each laboratory with a lab partner and you may acquire your data together during your scheduled lab time. However, you must NOT work with your lab partner beyond this. All analysis of data and calculations as well as all laboratory reports must be done on an individual basis. Failure to do so will result in a zero for the lab in question.

\noindent Please follow all safety procedures, especially by wearing safety glasses or goggles. When leaving the lab, please make sure it is in the same condition as it was when you arrived. Be respectful of others.

\paragraph{Laboratory Risk:}
Chemical exposure is a constant risk in a chemistry lab. To minimize the risk to yourself and those around you, the following rules must be followed:
\begin{itemize}
	\item Never taste or smell a chemical or pipette by mouth.
	\item Wash your hands before leaving the lab and frequently during the lab to avoid accidental contamination of yourself and others.
	\item Dispose of chemicals only as directed. Nothing goes down the sink unless expressly directed.
	\item Keep your work area clean; wipe up any spills (liquid or solid) immediately.
	\item Replace caps on reagent bottles, and never return chemicals to the original container.
	\item No shorts, tank tops, or sandals allowed in lab, and long hair should be restrained.
	\item Wear safety glasses at all times when in the lab.
\end{itemize}
Students enrolling in this course should realize that they are voluntarily exposing themselves to a variety of chemicals, some of which could be irritating or hazardous with excessive exposure.  For those persons with a sensitive medical condition such as allergies, precautions such as wearing additional protective garments, delaying enrolling, or even not enrolling in a class may be necessary.

\section*{Tentative Schedule}
This class will meet in room 208 of the Skaggs Center for Health \& Molecular Sciences (SCA)
\begin{itemize}
	\item Section 04 will meet on Tuesdays from 8:00 – 9:50 am
	\item Section 03 will meet on Fridays from 8:00 – 9:50 am
\end{itemize}

\paragraph{Week 1: Jan. 10 -- Jan. 14}~

\textbf{No lab this week!}

\paragraph{Week 2: Jan. 17 -- Jan. 21}~

\textbf{No lab this week!}

\paragraph{Week 3: Jan. 24 -- Jan. 28}~

Check-in and Safety and Techniques

\paragraph{Week 4: Jan. 31 -- Feb. 4}~

Density and Measurements

\paragraph{Week 5: Feb. 7 -- Feb. 11}~

Atomic Identification

\paragraph{Week 6: Feb. 14 -- Feb. 18}~

Chemical Formulas

\paragraph{Week 7: Feb. 21 -- Feb. 25}~

Chemical Names and Structures

\paragraph{Week 8: Feb. 28 -- Mar. 4}~

\textbf{No lab this week -- Spring Break!}

\paragraph{Week 9: Mar. 7 -- Mar. 11}~

Chemical Reactions

\paragraph{Week 10: Mar. 14 -- Mar. 18}~

Stoichiometry

\paragraph{Week 11: Mar. 21 -- Mar. 25}~

Chemical Equilibrium

\paragraph{Week 12: Mar. 28 -- Apr. 1}~

Collection of Hydrogen Gas

\paragraph{Week 13: Apr. 4 -- Apr. 8}~

Electrolytes

\paragraph{Week 14: Apr. 11 -- Apr. 15}~

Acid-Base Chemistry

\paragraph{Week 15: Apr. 18 -- Apr. 22}~

Lab Checkout and \textbf{Final Exam} (More details about the final will come)

\paragraph{Finals Week}~

No Final -- You took it last week!

\section*{Course Requirements}
Grades will be based on the following items:
\begin{description}
	\item[Pre-Lab Assignments] 10 Points Each
	\item[Safety and Clean-up] 5 Points Each
	\item[Lab Reports] 30 Points Each
	\item[Final Exam] 150 Points
\end{description}
Final Grades will be assigned according to the following grade scale:

\begin{tabular}{cl|c|cl}
	Percentage & Grade &  & Percentage & Grade \\ \midrule
	100--93.0  & A     &  & 77.0--73.0 & C     \\
	93.0--90.0 & A-    &  & 73.0--70.0 & C-    \\
	90.0--87.0 & B+    &  & 70.0--67.0 & D+    \\
	87.0--83.0 & B     &  & 67.0--63.0 & D     \\
	83.0--80.0 & B-    &  & 63.0--60.0 & D-    \\
	80.0--77.0 & C+    &  & < 60.0     & F
\end{tabular}
\paragraph{Pre-Lab Assignments:}
These assignments are in the laboratory manual and will be collected at the \emph{start} of lab each week.

\paragraph{Safety and Clean-up:}
Any student who violates laboratory rules or engages in unsafe behavior in the laboratory may lose points. Any student who leaves their station without fully cleaning up after the lab period will likewise lose points.

\paragraph{Lab Reports:}
Lab report pages are included at the end of the instructional material for each lab. These reports are due at the beginning of the next scheduled laboratory day. {\bfseries Missing more than 2 labs will result in a failing grade, regardless of the scores on all other assignments!}

\paragraph{Final Exam:}
The final exam is comprehensive and questions will draw on chemical concepts, laboratory techniques, results, and analysis and interpretation.

\paragraph{Attendance Policy:}
Students are expected to attend class. If you must miss class, contact the instructor ahead of time and arrange to attend another section, if possible.

\paragraph{Late Work Policy:}
All pre-lab assignments must be completed before the start of each lab period, and all reports are to be turned in on the day of the next lab period. Late work will not be accepted.

\paragraph{Make-up Work Policy:}
In general, there will be no opportunity to make up missed work unless arrangements are made ahead of time. If you must miss class, please contact the instructor ahead of time.

\section*{Miscellany}

\paragraph{Scientific Calculator:}
There are many different ways to calculate figures during homework. It is tempting to rely on Online resources such as \href{http://www.wolframalpha.com}{http://www.wolframalpha.com}, or to simply use a calculator application on a smart phone. During exams, however, any devices capable of connecting to the Internet will \emph{not} be allowed. You will instead need a scientific calculator capable of performing exponentiation and logarithms for the exams. Using this calculator exclusively while doing homework will ensure that you are familiar with it for use during exams.

\paragraph{Academic Integrity:}
Scholastic dishonesty will not be tolerated and will be prosecuted to the fullest extent (see \href{https://www.suu.edu/policies/06/33.html}{SUU Policy 6.33}). You are expected to have read and understood the current SUU student conduct code (\href{https://www.suu.edu/policies/11/02.html}{SUU Policy 11.2}) regarding student responsibilities and rights, the intellectual property policy (\href{https://www.suu.edu/policies/05/52.html}{SUU Policy 5.52}), information about procedures, and what constitutes acceptable behavior.

\paragraph{Mental Health:}
Mental and physical health are equal components to a holistic view of wellness and human thriving. Mental health should not be ignored, dismissed, or demeaned. If you find yourself struggling with mental health please visit \href{https://www.suu.edu/mentalhealth}{https://www.suu.edu/mentalhealth} for resources. There is also a link prominently on the right side of every Canvas page.

\paragraph{ADA Policy:}
Students with medical, psychological, learning, or other disabilities desiring academic adjustments, accommodations, or auxiliary aids will need to contact the Southern Utah University Coordinator of Services for Students with Disabilities (SSD), in Room 206F of the Sharwan Smith Center or phone (435) 865-8022. SSD determines eligibility for and authorizes the provision of services.

\paragraph{Emergency Management Statement:}
In case of emergency, the university's Emergency Notification System (ENS) will be activated. Students are encouraged to maintain updated contact information using the link on the homepage of the \emph{mySUU} portal. In addition, students are encouraged to familiarize themselves with the Emergency Response Protocols posted in each classroom. Detailed information about the university's emergency management plan can be found at: \href{http://www.suu.edu/emergency}{http://www.suu.edu/emergency}

\paragraph{HEOA Compliance Statement:}
The sharing of copyrighted material through peer-to- peer (P2P) file sharing, except as provided under U.S. copyright law, is prohibited by law. Detailed information can be found at: \href{https://help.suu.edu/article/1097/p2p-and-copyright-infringement}{https://help.suu.edu/article/1097/p2p-and-copyright-infringement}

\paragraph{LINK Statement:}
SUU faculty and staff care about the success of our students. In addition to your professor, numerous services are available to assist you with the achievement of your educational goals. SUU's LINK system may be used by faculty to notify you and/or your advisors of their concern for your progress and provide references to campus services as appropriate.

\paragraph{SUUSA Statement:}
As a student at SUU, you have representation from the SUU Student Association (SUUSA) which advocates for student interests and helps work as a liaison between the students and the university administration. You can submit My SUU Voice feedback by going here: \href{https://www.suu.edu/suusa/voice}{https://www.suu.edu/suusa/voice} Likewise, you can learn more about SUUSA's Executive Council here (\href{https://www.suu.edu/suusa/executive-council/}{https://www.suu.edu/suusa/executive-council/}) and about individual SUUSA's Student Senators here (\href{https://www.suu.edu/suusa/senate/}{https://www.suu.edu/suusa/senate/})

\paragraph{University Policies and Recommendations Regarding COVID-19:}
Southern Utah University has compiled a collection of information, policies, and recommendations related to COVID-19 at \href{https://www.suu.edu/coronavirus/}{suu.edu/coronavirus/}

\noindent I dearly want this semester to go smoothly vis-\`a-vis COVID-19, and I assume you all do as well. Toward that end, I encourage you all to exercise all reasonable precaution to prevent the spread of the coronavirus. This includes using the testing and self-reporting resources at the link above.

\noindent It may interest some of you to know that, for my part, I have been vaccinated with two doses of the Moderna vaccine, and more recently received a Pfizer booster. I will wear a mask on campus when appropriate. I will \emph{not} be wearing a mask as I lecture, since clear communication is my primary goal in the classroom.

\paragraph{Land Acknowledgement Statement}
SUU wishes to acknowledge and honor the Indigenous communities of this region as original possessors, stewards, and inhabitants of this Too’veep (land), and recognize that the University is situated on the traditional homelands of the Nung’wu (Southern Paiute People). We recognize that these lands have deeply rooted spiritual, cultural, and historical significance to the Southern Paiutes. We offer gratitude for the land itself, for the collaborative and resilient nature of the Southern Paiute people, and for the continuous opportunity to study, learn, work, and build community on their homelands here today. Consistent with the University's ongoing commitment to equity, diversity, and inclusion, SUU works towards building meaningful relationships with Native Nations and Indigenous communities through academic pursuits, partnerships, historical recognitions, community service, and student success efforts.

\paragraph{Disclaimer:}
Information contained in this syllabus, other than the grading, late assignments, make up work and attendance policies, may be subject to change as deemed appropriate by the instructor.

\end{document}
